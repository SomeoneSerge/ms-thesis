\chapter{Inverse Graphics} \label{chap:graphics}

Much of our work, while often purely geometric and abstract, is motivated by
the problem of reconstructing photorealistic models of real-world scenes.
There's no way we can give a thorough and self-contained treatise of the subject
here. We shall, however, outline some of the important directions.

It would be a naive endeavour to tackle the inverse problem without handling
the forward one. As we are talking about photorealistic imaging, some of the
questions to be concerned with is: what even to measure?  A good overview is
Naty Hoffman's SIGGRAPH tutorial~\cite{natyHoffmanPBS}. After introduction of a
reasonable approximation of Maxwell's law -- the Kajiya's ``Rendering
Equation''~\cite{kajiya1986rendering} -- the next problem is to find the
solutions to a Boltzmann-type integral equation.
The classic approach, taken by~\cite{kajiya1986rendering},
is to use Monte-Carlo methods, which gives rise to variety of raytracing algorithms.
Another common approach is the finite element method, for an overview of which
one could probabily refer to~\cite{lehtinen2004foundations}. Less common in
graphics community, but possibly still interesting alternative is also the boundary element
method~\cite{zhang2013fast}.

A subject of its own are models of light interaction -- the \( \rho \) function
appearing in Kajiya's equation. Related links and informal descriptions
can be found
in~\cite{hackmdRendering,hackmd3d,hackmdBeyondLambert,hackmdBlinn,hackmdDRM,hackmdTorranceSparrow,hackmdPhong}.

With this perspective, the physically-based rendering is naturally
differentiable (at least in the weak sense). \citet{tzuMaoEdgesampling} propose
direct differentiation of edge raytracing procedure by treating the interior
and the boundary of a rendered triangle individually and noting that for the
former one needs to simply differentiate the integrand, and the latter appears
in the form of uniform measures concentrated on the edges (kind of similar to
``delta-functions'') and can be handled by sampling edges explicitly.
Also of interest is Tzu Mao Li's thesis~\cite{tzuMaoThesis}.
\citet{mitsuba2,mitsuba2Math} take on different approach and suggest to
reparameterize the integral so that discontinuities disappear.

As inverse graphics is an extremely ill-posed problem with no single solution
that we could arrive at with straightforward gradient descent, we means of
incorporating our prior knowledge of the real world into the models to recover
solutions that do actually make sense. Specifically, we may want to pose
optimization problem that doesn't actually compare the physically-based render
of the scene against the reference: because e.g. occlusions could trap
reconstructed model in a local minimum. There are different approaches to
differentiable rendering, including \citet{softras,dibR} and many more.
One could also consult a reasonable collection~\cite{kaolin}

Recently has emerged the neural point-based graphics:~\citet{npbg}.
See~\citet{neuralRenderingOverview} for overivew of neural methods.

On another side of the spectrum are classic methods. First, for dealing with
geometry in presense of depthmaps, there's an family of methods inspired by
volumetric fusion of~\citet{sdfFusion}. The method allows to ``integrate''
multiple partial and noisy observations of the same surface into a single
model.  Model also has a reasonable Bayesian interpretation.
\citet{kinectfusion} further introduce the real-time KinectFusion algorithm
(and are, to our limited knowledge, first to use TSDF instead of SDF).
\citet{dynamicfusion} generalizes the method to dynamic setup
where the surface may be subject to non-rigid deformations over time.
They handle the problem by computing a motion field at each time frame
which ``reverses'' the deformation, restoring a ``canonical'' pose
of the object. In canonical pose the observations can be integrated using
the classical algorithm, and then integrated models can be put into
any given frame's pose using the same motion field.
Discussed so far are template-free methods that deal with arbitrarily deforming
surfaces. For certain tasks even more informative prior is required.
Thus, \citet{doublefusion}, dealing with human performances, propose to combine
template-based models (e.g.~\citet{smpl,softsmpl}) and dynamic volumetric
fusion.

Given a reconstructed coarse geometry (whether produced by volumetric fusion or
otherwise), one could use lighting information to restore finer details, e.g.
using photometric methods~\citet{flashmob}. Just thus~\citet{shadingRefinedSDF}
use shading information to refine volumetric models.

A curious development of volumetric fusion is the Neural Radiance Fields
by~\citet{nerf}. It sort of suggests to integrate light's interaction with the
scene, rather than depthmaps.  Specifically, it suggests to estimate the scene
as if it were a dense colored fog, through which a casted ray passes without
refractions and reflections, accumulating (or rather, losing) the view
direction-dependent color along the way. A colleague of mine, Alexey Boyko has
well described this as ``colored multipole fog''. The required underlying
datastructure then is a differentiable parameterized black-box function that
takes in a point and direction, and spits out absorbtion coefficient plus color
-- a task for a neural network!
There's a number of ways one could proceed with NeRF. One could seek to improve
its performance, e.g. by using a different blackbox, like a tensor train.  On
the other hand, just as TSDF-Fusion develops into DynamicFusion, one could try
to add a temporal dimension to NeRF (must be a decent exercise in Neurobayesian
methods).

The discussed methods in fact rely on more low-level machinery. For instance,
structure from motion, or efficient template fitting, all rely on extraction of
points of interest~\cite{r2d2} from different images of the same surface, and
the following synchronization of correspondences~\cite{birdalCorrespondences}.
Such tasks could potentially benefit from introduction of \emph{hyperbolic}
descriptors -- but we're yet to explore this direction. Aside from that, we
note that~\citet{birdalCorrespondences} relies on Riemannian optimization:
the model is implemented in geoopt~\cite{geoopt}.

One question which we hadn't time to touch upon and leave for the future is
that of software design (as in: architecture) for graphics.
Mostly, one has to figure out what design decisions there are to make,
what problems to run into. I made no progress on this, but here are some
possibly relevant links:
\begin{itemize}
\item There's CG course read by the Founding Fathers, \citet{blinnCatmullCG}.
\item Jim Blinn also has the great ``Jim Blinn's corner'' series:
\citet{jimTripPipeline,jimDirtyPixels,jimDirtyPixels}
\item \citet{sg13ApiFailed} tells the story of (still non-existing) standard
graphics in C++ and teaches a good lesson about evolution of APIs, time
constraints, trust, and agile.
\item \citet{shirleyRTWeekend} provides a beautiful introductory series on
raytracing, that starts from a scratch and describes the concepts in simplest
terms possible, without the burden of industry's legacy terminology and
pre-made design choices. A good place to start and see the subjects plain
without the overwhelming noisy background.
\item The excellent PBR-Book of \citet{pbrBook} is a a literate
programming-style book abot physically-based rendering techniques with industry
standards-compliant implementations in C++ and gives sense as to what sort of
challenges and decisions one has to face when designing software for compute
graphics. Specifically, when deciding about camera models or the form of
generic BRDF -- this book provides some background about what people out there
are already using. This is apart from being a really great and accessible book
about its subject. Especially note ``further reading'' page~\cite{pbrBookReading}.
\emph{Aside: it could prove a fun experience to try pair-programming pbrt in
e.g. Rust}.
\item \cite{shirleyKimTeaching} is a relevant twitter thread started by Shirley
and Kim -- contains a number of useful links on ``how to \emph{teach} computer
graphics?''.
\item One, short, of such references is \citet{texnh}.
\item Of obvious importance is ``Graphics Gems'' series:~\citet{graphicsGems}.
\item Another great series is realtimerendering:~\citet{RTR4}.
\item Realtimerendering's's page on intersection procedures has a number of
relevant links, including comprehensive books on the subject:~\citet{rtrIntersections}.
\item More gems are about realtime \emph{ray tracing}~\citet{rtGems}.
\item Further, NVIDIA has a series of blog posts on raytracing
essentials~\cite{rtEssentials}.
\item \citet{yoctoGL} is one of the attempts of making a non-object oriented
(and rather functional-style) graphics library
\item More on non-object oriented approaches, of interest
are~\citet{functional3DGraphics,elliottTBAG}, and of course numerous works
of~\citet{ActiveVRML,elliottAnimations}, which one could say served as an
inspiration for entire discipline. Notably, Elm by~\citet{elm} takes root in
these works. Some of the relevant examples of relevant implementations
of these ideas for 3D graphics include~\citet{muncheongFunctionalGames}.
\item Of interest are works of game developer~\citet{carpLang}, including his
``static lisp'' -- the Carp language.
\item \citet{levienPrefixSum} has a very curious blog on various things related
to GPU, including GPU Compute.
\item Already mentioned HyperRogue~\cite{hyperrogue} could serve as great a
reference for visualizations of hyperbolic spaces, despite its monolithic
structure.
\item Carl De Boor -- another Founding Father for computer graphics, the student of Isaac Jacob
``father of splines'' Schoenberg, demonstrates an excellent functional
language~\cite{deboorSmoothingBspl}
in his (often very applied) works on B-Splines and related. De Boor has a web-page
on which one could find links to his excellent lectures:~\cite{deboorPage}
\end{itemize}

This list barely scratches the surface, but provides some starting points
(...even if for myself).
