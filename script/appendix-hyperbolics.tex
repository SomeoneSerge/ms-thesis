\chapter{Hyperbolic representations}\label{appendix:hyperbolics}

What is the position of Gyrovector formalism, Einstein midpoints, and, often
set-free, hyperbolic geometry as relative to the framework of differential
geometry?

We first introduce the Hyperbolic space as the upper sheet of a hyperboloid in
Minkowski space. Then we show how the upper-sheet can be projected into
Poincar\'e ball model (with inherited metric) and further to upper-halfspace.
Until this point, we only discuss underlying sets and local differential
properties, such as local metric. To evaluate synthetic properties of the
Hyperbolic space, e.g. to compute distances between points, we then switch
between the available models arbitrarily -- they are isometric by construction,
while different tasks are easier accomplished in different models.

\section*{Hyperboloid and Poincar\'e disk models}

It seems to be a reasonably motivated to introduce a hyperbolic metric by
starting with the Hyperboloid model. For instance, we'll take on intuition
of~\cite{thurstonThree} and set out to construct a ``sphere'' of constant
negative curvature \( \frac{1}{r^2} = -1 \), i.e. ``of imaginary radius \( r=i
\)''.

Consider the Lorentz (Minkowski) space -- the \( (\mathbb{R}^{n+1}, Q) \),
consisting of points \( (x_1, \ldots, x_n, t) \) (conveniently denoted \( (x,
t) \)), endowed with a non-degenrate symmetric bilinear form
\[ Q((x, t), (x', t')) = \sum_{k=1}^n x_k x_k' - tt', \]
and corresponding pseudo-Riemannian metric \[ g_H = \operatorname{d}\mathrm{x}_1^2 +
\cdots + \operatorname{d}\mathrm{x}_n^2 - \operatorname{d}\mathrm{t}^2, \]
where \( \operatorname{d}\mathrm{x}_j \) is the chart-induced basis covector, as
introduced in previous appendix (we use straight characters \( \mathrm{x}_i \),
as opposed to \( x_i \) to denote the chart component \emph{functions}; in this
case chart is the identity and chart components are projections).

The underlying set of our model space is the upper sheet of the \(n\)-dimensional
Hyperboloid:

\begin{align*}
H &= \left\{ (x, t) \in \mathbb{R}^{n+1} \left| Q((x,t), (x,t)) = -1,~ t>0 \right.\right\}\\
&= \left\{ (x, t) \in \mathbb{R}^{n+1} \left| \sum_{k=1}^n x_k^2 - t^2 = -1,~t>0 \right.\right\}.
\end{align*}

The upper-sheet is a metric manifold, with global chart \( x \mapsto (x,
\sqrt{1 + \|x\|_2^2}) \), and metric induced from ambient Minkowski space. This
pull-back metric is a Riemannian (positive-definite) metric. However, instead
of investigating this metric on the Hyperboloid, we will straight away consider
the Stereographic projection that maps upper-sheet onto the \emph{Poincar\'e
ball} \[ \mathcal{P} = \{ p\in\mathbb{R}^n\left|~\|p\|_2^2 \leq 1\right.\}. \]
This map is realized as follows: a point \( (x, t) \) on the upper-sheet is
projected onto the horizontal plane \( \{ (y, 0) \left| y\in
\mathbb{R}^{n}\right. \} \) with the center of projection at the vertex of the
lower-sheet, i.e. \( (0, -1) \). One can easily derive that the projection is \(
(x, t) \mapsto \frac{1}{1+t}x: H\to\mathcal{P} \), and has the inverse \(
\phi:\mathcal{P}\to H, \) \[ \phi(p) = (\frac{2p}{1 -
\|p\|_2^2},~\frac{1+\|p\|_2^2}{1-\|p\|_2^2}). \]

The Poincar\'e ball has a very natural global chart -- the identity map \(
\mathrm{p}=p\mapsto p \), which induces a convenient basis of partial
derivatives,
\( \operatorname{id}_*\partial_i\simeq\partial_i,~i=\overline{1,n}, \) in the
tangent bundle \( \mathcal{T}\mathcal{P} \). The projection inverse \( \phi \)
allows to pull-back the Minkowski metric inherited by upper-sheet onto
Poincar\'e disk: \[ (\operatorname{id}_{H\to(\mathbb{R}^{n+1},g_H)}\circ
\phi)^* g_H = \phi^*g_H. \]
It's useful to describe this metric in chart-induced
coordinates. Specifically, we need to compute
\[ g_H(\phi_* \partial_i,~\phi_* \partial_j). \]

A tangent vector \( \phi_*\partial_i \in \mathcal{T}\mathbb{R}^{n+1} \) is
naturally described by its action on a scalar function \(
f:\mathbb{R}^{n+1}\to\mathbb{R} \), so
\begin{align*}
(\phi_* \left.\partial_i\right|_p)f &= \left.\partial_i\right|_p(f\circ \phi)\\
&= \sum_{k=1}^{n+1} (\left.\partial_k\right|_{\phi(p)} f) (\left.\partial_i\right|_p \phi_k)\\
&= \left(\sum_{k=1}^{n+1} (\partial_i \phi_k)(p) \cdot
\left.\partial_k\right|_{\phi(p)}\right) f,
\end{align*}
where we, slightly abusing notation, treat partial derivatives as
``polymorphic'', so that the ``type'' of each instance is inferred from types
of their arguments (\( f \) or \( \phi_k \)).
Further, the coefficients of basis vectors
\begin{align*}
(\partial_i\phi_k)(p)
&= \frac{2p_i}{(1-\|p\|_2^2)^2} \sum_{k=1}^{n+1} 2p_k \partial_k
+ \frac{2}{1 - \|p\|_2^2} \partial_i
+ \frac{2p_i}{1 - \|p\|_2^2} \left( \frac{1+\|p\|_2^2}{1-\|p\|_2^2} + 1\right) \partial_{n+1}\\
&= \frac{4p_i}{(1-\|p\|_2^2)^2} (\partial_{n+1} + \sum_{k=1}^{n+1} p_k \partial_k)
+ \frac{2}{1-\|p\|_2^2}\partial_i.
\end{align*}

Now we can compute the coefficients of the metric tensor.
First, for \( i\neq j \):
\begin{align*}
g_H(\phi_*\partial_i, \phi_*\partial_j)
= &~\frac{4p_i 4p_j}{(1 - \|p\|_2^2)^4}
     g_H(\partial_{n+1} + \sum_{k=1}^n p_k \partial_k,~
         \partial_{n+1} + \sum_{k=1}^n p_k \partial_k) \\
  &~+ \frac{4}{(1-\|p\|_2^2)^2} g_H(\partial_i,~\partial_j) \\
  &~+ \frac{8p_i}{(1 - \|p\|_2^2)^3} g_H(\partial_{n+1} + \sum_{k}\partial_k,~\partial_j) \\
  &~+ \frac{8p_j}{(1 - \|p\|_2^2)^3} g_H(\partial_i,~\partial_{n+1} + \sum_{k}\partial_k) \\
= &~(16p_ip_j)\frac{-1 + \|p\|_2^2}{(1 - \|p\|_2^2)^4} \\
  &~+ 0 \\
  &~+ \frac{8p_i}{(1 - \|p\|_2^2)^3} g_H(p_j \partial_j,~\partial_j) \\
  &~+ \frac{8p_j}{(1 - \|p\|_2^2)^3} g_H(\partial_j,~p_i \partial_i) \\
= &~0.
\end{align*}

Finally,
\begin{align*}
g_H(\phi_*\partial_i,~\phi_*\partial_i)
= &~\frac{-16 p_i^2}{(1 - \|p\|_2^2)^3} 
  + \frac{4}{(1-\|p\|_2^2)^2} 
  + \frac{16p_i^2}{(1 - \|p\|_2^2)^3} \\
= &~\frac{4}{(1 - \|p\|_2^2)^2}.
\end{align*}

Thus
\[ \left.\phi^*\right|_p g_H = \frac{4}{(1 - \|p\|_2^2)^2} \operatorname{d}\mathrm{p}^2. \]

This is the Poincar\'e metric usually given in papers just by formula, only now
we've made sure that this metric is \emph{indeed} the pullback of Hyperboloid's
metric under the central projection, and the two models of hyperbolic geometry
are isometric by-construction. This metric is conformal (preserves angles), as
it's a multiple of the ambient Euclidean local metric \( \operatorname{d}\mathrm{p}^2
\).

\section*{Poincar\'e upper halfplane}

Obtained from the Ball via inversion in a sphere.

Results in underlying set \( \{ x\in\mathbb{R}^n \left|~x_n > 0\right.\} \)
and the pull-back metric \( \frac{1}{x_n^2} \operatorname{d}\mathrm{x} \).

Distance of a vertical segment can be easily integrated...

Geodesic between points on the same vertical line is exactly the vertical line segment...

M\"obius transformations are isometries...

One can move any two points onto same vertical line via M\"obius transformation
and then compute the distance...

Stereographic projection is also a simple M\"obius transformation, specifically
an inverison in a sphere (plus throwing away a coordinate)...

\section*{Other models}

A different projection could give the Klein model where geodesics are straight
Euclidean lines. Another interesting projection is the Band model which allows
to make a chosen curve appear as a straight line. Very motivating overviews
are~\citet{bulatovConformal} and everything related to the amazing
\texttt{HyperRogue}~\cite{hyperrogue} project.

\section*{Group of isometries of hyperbolic plane}

We've already mentioned the M\"obius (Lie) group...
Please refer to~\cite{beardonGeometryDiscrete,hubbardTeichmuller}

\section*{Tilings}

As \citet{yaSaTilingBased} demonstrated, tilings and discrete groups acting on
Hyperbolic spaces are of utmost importance. An entry point is~\citet{gromov}.


\section*{Harmonic analysis}

For our further work on ``hyperbolic convolutions'' familiarity with harmonic
analysis and gauge theory is required. \nameref{sec:history} provides
a number of relevant links.
\citet{stollharmonic} discusses PDEs on real Hyperbolic space, and e.g.
describes solutions to Laplace equation (the spherical harmonics).

\section*{Barycentres vs Einstein midpoints}

Averaging in metric spaces relates to the notion of a barycentre of a measure:
a point \( b \) of a metric space \( (M, \rho) \) is a barycentre of a measure
\( \mu \) if it minimizes the variance functional
\[ b \mapsto \int \rho(b, \cdot)^2\operatorname{d}\mu. \]

An average or mean of \( n \) points, say \( p_1, \ldots, p_n \), is a
barycentre of the uniform measure concentrated on those points, i.e. barycentre
of \( \frac1n\sum_{i=1}^n \delta_{p_i} \).

The Einstein's midpoint, used e.g. in~\cite{khrulkov} is not
the\footnote{``the'', as in negatively-curved spaces barycentres are unique}
variance minimizer for Hyperbolic space. Instead, ``Einstein's midpoint'', or
Einstein's \emph{average} of relativistic velocities \( v_1, \ldots, v_n \)
(relative to a chosen rest frame) of particles with invariant masses \( m_1,
\ldots, m_n \), is the velocity of the center of momentum frame of
this system. To give sense of what functional ``midpoint'' minimizes, the
center of momentum-frame is the frame in which total energy of the system is
minimized, or, equivalently, one in which perceived total momentum vanishes.

\citet{diffThroughFrechet} propose a differentiable iterative algorithm
that computes the true mean. See also \citet{gdFrechetHyperbolic}.
