\chapter{Hyperbolic convolutions for image embeddings} \label{appendix:hconv}

\begin{figure}[ht]\center
\begin{tikzpicture}[scale=0.5, every node/.append style={transform shape}]
\node at (0,0) {
   \input{art/image-convolutions-2019.pdf_tex}
};
\end{tikzpicture}
\caption{A visualization of ``hyperbolic'' CNN for representation learning}
\end{figure}

In \emph{very} short: we (Max Kochurov, Rasoul Karimov, Maria Taktasheva,
Cyrill Mazur, Serge Kozlukov) attempt to construct ``convolutional'' layers
which ``consume'' (multidimensional) arrays of \emph{points on manifolds} (each
point interpreted as a ``descriptor of a pixel''). Our idea is that hyperbolic
space is a continuous analogue of a tree and our convolutional operation
resembles a ``decisison rule for walking down the tree'', choosing new
directions based on descriptors of neighbour pixels. We also generalize Batch
Normalization to such setting be appealing to the notion of variance and
(variance-minimizing) barycenter of a measure in a metric space. Results are
not satisfying, so we do not discuss the model in detail. The general idea
however is relevant to the rest of the work.
