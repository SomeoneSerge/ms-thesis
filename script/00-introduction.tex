\chapter{Introduction}

\epigraphhead[50]{%
    \epigraph{"So yeah, thought ya\\
    Might like to go to the show.\\
    I've got some bad news for you, Sunshine -- Pink isn't well!"
    }{Pink Floyd, "The Wall"}
}

\section*{For Whom and What?}

What makes a good thesis? There are 127 members in the Telegram chat of
"Skoltech Datascience 2018". After they've submitted their great theses,
received feedback from their reviewers, successfully passed the defences and
graduated -- how many people are to read their work then? As these 127 students
are briliant, I dare assume they must've also published their results at top
conferences such as NeurIPS, prior to finishing their degrees. Thus at the
least, their work must be soon read and cited by fellow researchers in their
respective fields.  That is at the \emph{very} least, until there's a new SOTA
method and all prior approaches get discarded, which in such active and
yet-un-shaped field as ``deep learning'' sometimes could happen even next
winter. One more link that could lead future generations to the works being
defended to-day is that these noble soon-to-be-masters will surely proceed in
their research, build on their current results something even more advanced.
That should definitely live longer than until next conference.  That being
said, many Masters will become Doctoral Students... yet I know in person those
who have been so exhausted by the constant race, frustration, and lack of
determinism, which comprise a necessary part of research process, that they
have already decided to leave Academia and go for the well-paid Industrial Job.
The Job with local and concrete goals.  The Job with visible and tangible
outcomes. The Job with predictable short-term rewards. 
One of my friends -- a most talented mathematician, far more able than me
-- began his Doctoral studies, but changed his mind within just half a year and
migrated to a different country to write frontend in JavaScript.
With all that in mind, what could make a thesis worth reading? Who is the
reader I shall have in mind as I write this? I believe it's the one who
struggles. I have read many PhD theses over these two years, and the part I
always enjoyed the most was ``Preface''. The ``Preface'' tends to contain a
personal story behind people's work. There I could see all the pain authors
went through along their ways. The mistakes they made. And everything that kept
them going!  Thus my idea is that I'm writing for someone who is desperately
trying to deliver the tiny little message, the one who makes bad choices, the
who in spite of failures and disapproval still means to keep walking the path
and engaging. I'm writing for future me.

Now that we've figured out \emph{whom} this writing is for, we should concern
ourselves with the question of \emph{what}. What message could be of use to The
Reader? What words could resonate with The Reader's heart and convince of
actual relevance to His\footnote{I have tried for many years to use "they and
them" as gender-neutral pronouns, yet I shall join~\cite{strictlyEnglish} in
that it is ``an abominable and I want no part in it''.  However, instead of
following the rule of ``masculine be taken to include feminine whenever
necessary'' I would rather strip ``masculine'' of any identity and declare it
the actual gender-less one, all throughout this thesis.} work?  At this point
I'd want to mention a little about my background, which hasn't much to do
directly with ``Data Science''. I did a Bachelor's degree in something called
``Applied Mathematics'', specifically in the instance that mostly meant
ordinary differential equations, optimal control, algorithms, and
``programming''. My Bachelor's scientific advisor -- Anatoly Baskakov -- was
however nothing like that! He would always proudly say ``I am and always have
been a Pure Mathematician!''.  Although I come not even close to comprehending
his ideas, I would best describe his main interest as ``linear functional
analysis'', showing that one could treat things like PDEs or Harmonic Analysis
in a much more convenient setting, such as Banach spaces and modules. He'd also
say ``it is in a way naive, to tackle the Nonlinear, when there's still so much
work to be done on the Linear''. I aspire one day to say I am a pure
mathematician too.  But now to finish the joke, before my Bachelor's I could
often reply to an arbitrary question with ``I wouldn't know, I'm a coder, not a
mathematician".  Yes, I started out obsessed with technologies, frameworks, and
``software design'', but unable to see connections to Mathematics. I arrived at
the point of ``I'd rather not bother writing the code, I'm more of a
mathematician''. What of relevance do these transitions imply about the author,
you might ask?  \emph{First} implication is that which I did \emph{not} have
when starting my MSc:  I didn't have any ``geometrical'' courses -- Metric and
Differential Geometry are crucial to my Master's work --, didn't have proper
Probability, Statistics, and Physics -- the core of ``Statistical Learning
Theory'' --, never had time to write a raytracer or rasterizer -- and lack of
familiarity with Computer Graphics community's conventions was a significant
impediment to my work --, nor did I have much of ``Foundations of Mathematics''
like Categories~\citep{categoriesWorking}, Logic, Types, Functional
Programming, Theorem Provers~\citep{coq,lean}, and whatever is called ``Model
Theory''~\citep{marker2006model} -- of huge importance on their own, these must
soon prove instrumental in fixing the lamentous state of Array
Programming~\citep{dexlang,mcdonellGPU} and managing Deep Learning
pipelines~\citep{mokhov2018build}.  \emph{Second} implication was painful
\emph{awareness} of my own lack of knowledge, and of the problems and
inconsistenceis, that people in ``Applied Sciences'' -- in ``Deep Learning'' in
particular -- tend to neglect and ignore.  For instance, a significant
researcher in Bayesian methods nowadays might easily fail to describe what ``a
derivative of a function\footnote{In the simple context of vector spaces} at a
point'' actually is. It has been my belief that this lack of \emph{conceptual}
understanding of underlying mathematical framework must be slowing down
development of our field. As you dig down this rabbit hole, more and more
questions pop up: the derivative at a point is a linear operator acting on
``increment vectors'', the second derivative acts on the operators themselves,
and so on, ``operators all the way down''. But the basic task of writing down
Taylor series becomes increasingly cumbersome routine -- is there no other
truth?!~\citep{elliott2018simple} What is a
Tensor~\citep{bradley2020interface}? Is it just an array of numbers or is it a
multilinear function? What's the fuzz with categories and universal property?
Have they anything to do with geometry?  What is this talk about contra- and
co-variance?  What are jets~\citep{betancourt2018geometric}?  Why on earth does
Arnold~\citep{arnoldPDEs} denote basis vectors with \( \frac{\partial}{\partial
x^i} \), as operators?  What is a
\emph{convolution}~\citep{feichtingerFAHA,cohen2018general,e2cnn} and what are
its defining properties? What is \emph{Hyperbolic Space}~\citep{ganeaHNNs}, and
how do \emph{convolutions} work there? What is \emph{curvature}? What does it
mean for \emph{data}~\citep{khrulkov} to have hidden underlying curved
geometry? Suddenly, what are actual quantities in Kajiya's Rendering
Equation~\citep{kajiya1986rendering} and how do they relate to Maxwell's? To
more ground questions, how does one
annotate~\citep{documentationResearch,struturingDocumentation,doctest,doctestRust,doctestCpp}
and statically check array shapes~\citep{rush2019tensor,rush2019tensor2} in
\texttt{NumPy} or \texttt{PyTorch}, and what has it got to do with
\emph{Dependent Types}~\citep{dexlang}? How does one ensure
reproducibility~\citep{inriaSacred,catalyst,pytorchLightning,xinLiangPipelines,},
manage dependencies and packaging, handle CI/CD in the environment, where
package sizes are measured in gigabytes, and job running times in days?  How
does one even write tests in such eco-system?  These are just tiny fraction of
the questions that would put me at an unease throughout my MSc. Questions of
this kind are either unknown to or deemed uninimportant by an average working
researcher in the ``Deep Learning''. I would imagine my Reader to be
unsatisfied with such state of affairs. If I were to compress my message for
you, Dear Reader, to bare minimum, it would be this: I am, too, trying to
figure this all out.

\section*{A condensed story of the last two years}

The thesis you have at your hands, Dear Reader, has the title of ``Geometric
Deep Learning for Inverse Graphics''. If my memory does not fail me, which it
lately has been, it is the third version of my work's title. The frontpage also
says, ``Supervised by Vladimir Spokoiny and Dmitry Ulyanov'', which is what I
reported to Skoltech's Education Office earlier last Fall. That too has been
constantly changing over the course of two years of Master's program.  In 2018,
as I endeavoured to work right at the edge of my beloved Mathematics and this
new ``AI'' field, I first started learning about optimal transportation theory.
Optimal Transportation~\citep{villaniOldNew,ambrosioOTSummerSchool} is a
beautiful mixture of what seems like pure math and geometry, but happens to
have a tremendeous amount of connections to Physics and Statistics -- through
partial differential equations, through probabilistic results, through games
and operations research, through optimization theory, and much more! I
connected with Thibaut Le Gouic, a Research Fellow at Higher School of
Economics, widely known for his work on Barycentres in Wasserstein
Spaces~\citep{le2017existence}. I started learning about Metric Geometry and
curvature. This has something to do with the sum of angles in a triangle, rates
of growth of volume, and existence of unique shortest paths between points -- I
shall touch upon that a little later in the thesis, as it will turn out to be
highly relevant to many other subjects. At about same time I was passing the
Large Scale Optimization course by Yury Maximov. Students of Datascience-2018
could tell many stories about the full month of sleepless nights, when we would
only live to submit homeworks and kickstart ADMM~\citep{admmTweet,admmBoyd}.
For me, however, it was first of all the time when we had to gather in teams to work
on course projects. I and fellow students chose to reproduce results
of~\citet{acceleratingNatgrad} and that was life-changing in two aspects: it
began my long and fruitful collaboration with now-my-friend Max Kochurov, and I
finally started learning Differential Geometry.
